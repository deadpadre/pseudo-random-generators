\documentclass[12pt, a4paper]{article}
\usepackage[utf8]{inputenc}
\usepackage[russian]{babel}
\usepackage{amsmath}
\usepackage{amssymb}
\usepackage[margin=1.0in]{geometry}

\newcommand{\dom}{Dom \;}
\newcommand{\uniform}{Un \;}
\newcommand{\bool}{\{0, 1\}}
\newcommand{\eps}{\varepsilon}

\begin{document}

\paragraph{Определение} Последовательность $\{\alpha_n\}$ называется пренебрежимо малой, если $\forall poly(n) \; \exists N : \; \forall n > N \; |\alpha_n| < \frac{1}{poly(n)}$.

\paragraph{Определение} $\{f_n\}, \; f_n : \bool^{l(n)} \rightarrow \bool^{m(n)}$ - семейство односторонних функций, если:

\begin{itemize}
 \item $\{f_n\}$ - полиномиально вычислимо относительно $n$;
 \item $\forall \{C_n\}$ - последовательности схем полиномиального размера
 \[
  P[C_n(f_n(x)) \in f_n^{-1}(f_n(x))] \sim 0
 \]
 \item  $\uniform(\dom f_n)$ - доступно.

\end{itemize}

\paragraph{Определение} $d(\alpha_n, \beta_n) = \sum\limits_{x \in \dom} \frac{|\alpha_n(x) - \beta_n(x)|}{2}$

\paragraph{Определение} Распределение $\mu_n$ называется полиномиально моделируемым, если существует (вероятностный) алгоритм $A$, получающий на вход $\uniform p(n)$ и $\forall x \in \dom \mu_n \; \; P[A = x] = \mu_n(x)$.

\paragraph{Определение} Распределение $\mu_n$ называется доступным, если существует полиномиально моделируемое распределение $\eta_n$ такое, что $d(\mu_n, \eta_n) \sim 0$.

\paragraph{Свойства}

\begin{itemize}
 \item[1)] $\alpha_n \sim \beta_n, \beta_n \sim \gamma_n \Rightarrow \alpha_n \sim \gamma_n$
 \item[2)] $\alpha_n \sim \beta_n$, и $\gamma_n$ независима от $\alpha_n$ и $\beta_n$. Тогда $\alpha_n\gamma_n \sim \beta_n\gamma_n$ (конкатенация).
 
 $\triangle$ Пусть это не так. Тогда существует $\{C_n\}$ полиномиального размера, такие, что
 $|P[C_n(\alpha_n\gamma_n)=1] - P[C_n(\beta_n\gamma_n)=1]|$ - не пренебрежимо малая последовательность. Заметим, что $|P[C_n(\alpha_n\gamma_n)=1] - P[C_n(\beta_n\gamma_n)=1]| \leq |E_{\gamma_n}(P[C_n(\alpha_n\gamma_n)=1] - P[C_n(\beta_n\gamma_n)=1])| \leq E_{\gamma_n}(|P[C_n(\alpha_n\gamma_n)=1] - P[C_n(\beta_n\gamma_n)=1]|) \leq |P[C_n(\alpha_n\gamma_{max})=1] - P[C_n(\beta_n\gamma_{max})=1])|$, то есть $\alpha_n \ncong \beta_n$. Противоречие.$\Box$
 \item[3)] Пусть $\{T_n\}$ - последовательность схем полиномиального размера, и $\alpha_n \sim \beta_n$. Тогда $T_n(\alpha_n) \sim T_n(\beta_n)$
\end{itemize}

\paragraph{Определение} $h_n(x)$ называется трудным битом для односторонней $f_n(x)$, если $h_n(x)$ полиномиально вычислима, и $\forall \{C_n\}$ - схем полиномиального размера $P[C_n(f_n(x))=h_n(x)] \sim \frac{1}{2}$.

\paragraph{Определение} Две последовательности $\alpha_n, \beta_n$ называются вычислимыми и неотличимыми, если $\forall \{C_n\}$ - схем полиномиального размера $P[C_n(\alpha_n)=1] \sim P[C_n(\beta_n)=1]$.

\paragraph{Лемма} Пусть $\{f_n\}$ - семейство односторонних функций, являющихся перестановками, а $\{h_n\}$ - ее трудный бит. Тогда
\[
 h_n(x)f_n(x) \sim r_n f_n(x) \sim r_n x
\]
где $r_n$ - чистый случайный бит.

$\triangle$ Докажем правую эквивалентность. Поскольку $P[C_n(x)=1]=P[C_n(f_n(x))=1]$, и используем свойство III.

Докажем левую эквивалентность. От противного. Пусть $\exists \{C_n\}$ - схем полиномиального размера таких, что $\exists s(n) = poly(n)$, и $\forall N \; \exists n > N$:
\[
 |P[C_n(h_n(x)f_n(x)) = 1] - P[C_n(r_n f_n(x)) = 1]| > \frac{1}{s(n)}
\]

Построим $\{R_n\}$:

\begin{itemize}
 \item[I] $R_n(r_n, f_n(x)) = r_n$, если $C_n(0 f_n(x)) = C_n(1 f_n(x))$;
 \item[II] $R_n(r_n, f_n(x)) = 0$, если $C_n(0 f_n(x)) = 1, C_n(1 f_n(x)) = 0$;
 \item[III] $R_n(r_n, f_n(x)) = 1$, если $C_n(1 f_n(x)) = 1, C_n(0 f_n(x)) = 0$;
\end{itemize}

Тогда легко проверить, что 

\[
 |R_n(r_n, x)| = |P[C_n(h_n(x)f_n(x)) = 1] - P[C_n(r_n f_n(x)) = 1]| > \frac{1}{s(n)}
\]

Это значит, что $\{R_n\}$ обращает функцию $f_n$. Противоречие. $\Box$

\paragraph{Определение (код Адамара)} Пусть $x \in \bool^n$. Тогда кодом Адамара $H = H_x : \bool^{2^n} \mapsto \bool$ называется функция: $\forall y \; H_x(y) = x \odot y$. Эта функция является линейной по $y$.

\paragraph{Лемма} Пусть $H_1, H_2$ - коды Адамара, и $H_1 \neq H_2$. Тогда $H_1$ отличается от $H_2$ ровно в половине точек.

$\triangle$ Исходя из линейности, <TODO>

\paragraph{Лемма (восстановление по испорченному коду Адамара)} Пусть имеется испорченный в не более чем $\eps$ доле точек код Адамара $\widetilde{H}$ (известно, что есть такое $x$, что $\widetilde{H}$ отличается от $H$ не более чем в $\eps$ доле точек, где $H = H_x$ - правильный код Адамара для $x$; при этом само $x$ не известно), $0 < \eps < \frac{1}{4}$. Тогда по $\widetilde{H}$ можно однозначно восстановить $x$.

$\triangle$ Восстановим правильное значение в точке $y$. Зафиксируем $y$. $H(y) = H(y + r) - H(r) \stackrel{2 \eps}{=} \widetilde{H}(y + r) - \widetilde{H}(r)$ (то есть последнее равенство верно с точностью до ошибки в $2 \eps$ доле всех $r$). Первое равенство - по линейности, последнее равенство следует из того, что максимальная ошибка при сложнеии или вычитании суммируется. Следовательно, $H(y) = \widetilde{H}(y + r) - \widetilde{H}(r)$ выполняется в $1 - 2\eps > \frac{1}{2}$ случаев (а что если всех возможных r очень мало, а 1 - 2 eps очень близко к 1/2 настолько, что r не может обеспеить точность??????), значит, $H(y)$ определяется точно, как наиболее часто встречающееся по всем возможным $r$ значение $\widetilde{H}(y + r) - \widetilde{H}(r)$. Определив значения H в базисе, восстановим $x$, имея проекции на все элементы базиса.
$\Box$

\paragraph{Лемма} Пусть события $A_1, \dots, A_N$ попарно независимы, и каждое происходит с вероятностью $\leq \frac{1}{2} - \eps$. Тогда $P[$произошло больше половины событий$] \leq \frac{1}{\eps^2 N}$
<TODO>

\paragraph{Теорема (восстановление списка по испорченному коду Адамара)} Существует вероятностный алгоритм, который по $n, \eps$ имея в качестве внешней процедуры $\widetilde{H}$, такой, что расстояние Хемминга (???) $(\widetilde{H}_x, H_x) \leq \frac{1}{2} - \eps$, за время $poly(n, \frac{1}{\eps})$ с вероятностью $ > \frac{1}{2}$ находит список длины $poly(n, \frac{1}{\eps})$, содержащий $x$. (список экспонентациальной длины, втф???)

\paragraph{Теорема (Левина-Голдрайха)} Пусть $f$ - односторонняя функция. Тогда $x \odot y$ - трудный бит для функции $[x, y] \rightarrow [f(x), y]$.

\paragraph{Определение (универсальное семейство хэш-функций)} Пусть $H$ - семейство функций (не обязательно всех) вида $\bool^n \rightarrow \bool^s$, и $h = \uniform H$. $H$ называется универсальным, если

\begin{itemize}
 \item[I] $\forall x \; h(x)$ равномерно распределено в $\bool^s$
 \item[II] $\forall x_1 \neq x_2 \;$ $h(x_1)$ и $h(x_2)$ - независимы, или, иначе говоря, пары вида $(x_1, x_2)$ равномерно распределены в $\bool^s$(??? иначе говоря а что с неравными иксами) 
\end{itemize}

Как пример можно привести линейные функции.

Далее еще будет важно, чтобы семейство задавалось полиномиальным количеством параметров (?) от $m, s$.

\paragraph{Определение (энтропия)} Пусть $\alpha$ - случайная велечина с $n$ значениями, и вероятности исходов - $p_1, \dots, p_n$. Энтропией называется:
$$
\begin{array}{rcccl} $Шеннона$: & H_0      & = & \sum p_i \log_2 \frac{1}{p_i} \\
                     $Ренье$:   & H_1      & = & \log_2 \frac{1}{\sum p_i^2} \\
               $минимальная:$   & H_{\infty} & = & \min_i log_2 \frac{1}{p_i}
\end{array}
$$
и верно соотношение $H_{\infty} \leq H_1 \leq H_0$. Вообще говоря, $2^{-H_r} = \sqrt[r]{\sum p_i^{r + 1}}$.

(без доказательства) Максимум всех энтропий при числе исходов $n$ достигается при равной вероятности всех исходов и равен $\log_2 n$.

$\triangle$ ($H_{\infty} \leq H_1$) <TODO>


\paragraph{Лемма (о сглаживании)} Пусть $H$ - универсальное семейство хэш-функций с параметрами $(m, s)$, $h = \uniform(H)$, $x$ - случайная велечина в $\bool^m$, $H_1(x) \geq k$, $r = \uniform(\bool^s)$ (!!!! почему и там и там s), $L_1(\alpha, \beta) = \sum\limits_y |P[\alpha=y] - P[\beta=y]|$.

Тогда

$$
 (h(x), h) \sim_{2^{\frac{s - k}{2}}} (r, h) 
$$
где $\sim$ понимается в смысле $L_1$ расстояния.

$\triangle$ Пусть $|H| = 2^l$. Одно из неравенств далее следует из того, что $E \xi^2 \geq (E \xi)^2$.

$$L_1 = \sum_{h, a} |2^{-l} P_x[h(x) = a] - 2^{-l-s}| \leq |E_{h, a} |P_x[h(x) = a] 2^s - 1| \leq $$ 

$$ \leq \sqrt{E_{h, a} (P_x[h(x) = a] 2^s - 1)^2} \leq
\sqrt{E_{h, a} (2^s \sum_x P(x) \mathbb{I}[h(x) = a] - 1)^2} \leq$$
$$\leq \sqrt{E_{h, a} (2^s \sum_{x_1} P(x_1) \mathbb{I}[h(x_1) = a] - 1)} \sqrt{E_{h, a} (2^s \sum_{x_2} P(x_2) \mathbb{I}[h(x_2) = a] - 1)} =$$

$$ =\sqrt{E_{h, a} (\sum\limits_{x_1, x_2} 2^{2s} P(x_1) P(x_2) \mathbb{I}[h(x_1) = h(x_2) = a]) + Q} = (*)$$

$Q$ - остаток, и $Q = E_{h, a}(1 - 2^{s + 1} \sum\limits_x P(x) \mathbb{I}[h(x)=a]) = 1 + (-2) = -1$.

$E_h(\mathbb{I}[h(x_1) = h(x_2) = a]) = 2^{-2s}$, если $x_1 \neq x_2$, и $2^{-s}$ в другом случае. Для того, чтобы посчитать сумму в (*), прибавим и вычтем этот член.

Из условия $H_1(x)  \geq k$ вытекает, что $\sum\limits_x P^2(x) \leq 2^{-k}$ (используется в последнем неравенстве).

$$(*) = \sqrt{1 - (\sum\limits_{(x, x)} P^2(x) - \sum\limits_{(x, x)} 2^{s} P^2(x) ) - 1} = 
\sqrt{\sum\limits_x (2^s - 1) P^2 (x)} =$$

$$=\sqrt{(2^s - 1) \sum\limits_x P^2(x)} \leq
\sqrt{2^s 2^{-k}} =
2^{\frac{s - k}{2}}$$

$\Box$

\paragraph{Лемма} Если $\alpha, \beta$ - независимые случайные велечины с конечным числом значений, то $H_1((\alpha, \beta)) = H_1(\alpha) + H_1(\beta)$ (вообще говоря, верно для любого количества велечин).

$\triangle$ Пусть $p_1, \dots, p_n$ - вероятности значений $\alpha$, $q_1, ..., q_m$ - вероятности значений $\beta$. Тогда вероятности значений $(\alpha, \beta)$ - $p_i q_j; 1 \leq i \leq n, 1 \leq j \leq m$. Очевидно, что
$$
2^{-H_1((\alpha, \beta))} = \sum\limits_{i, j} (p_i q_j)^2 = \sum\limits_{i, j} p_i^2 q_j^2 =
(\sum\limits_{i} p_i^2)(\sum\limits_{j} q_j^2) = 2^{-H_1(\alpha)} 2^{-H_1(\beta)}
$$
из чего следует требуемое.
$\Box$

\paragraph{Лемма} Пусть $f$ - односторонняя функция, $b$ - ее сложный бит, $r_1, \dots r_n$ - чисто случайные биты. Тогда
$$f(x_1) \dots f(x_n) b(x_1) \dots b(x_n) \sim f(x_1) \dots f(x_n) r_1 \dots r_n$$

$\triangle$ Будем заменять по одному сложному биту на случайные, и именно, докажем, что 
$$f(x_1) \dots f(x_n) b(x_1) \dots b(x_{k - 1}) b(x_{k}) r_{k + 1} \dots r_{n} \sim f(x_1) \dots f(x_n) b(x_1) \dots b(x_{k - 1}) r_{k} r_{k + 1} \dots r_{n}$$

Имеем
$$f(x_k)b(x_k) \sim f(x_k) r_{k}$$

Видно, что, кроме этого, в больших выражениях все одинаковое, значит, из маленкой эквивалентности следует большая, так как можно дописать в нужные места одинаковые части (по свойству III получившиеся выражения будут эквивалентны). Далее доказательство по индукции.

$\Box$


\paragraph{Теорема} Пусть существует односторонняя функция $f:\bool^n \rightarrow \bool^m$, $H_1(f(\uniform_n)) = n - c, \; 0 < c < 1$. Тогда существует PRG.

$\triangle$ Будем считать, что у $f$ есть трудный бит $b$. Пусть также имеется случайная хэш - функция из универсального семейства $h : \bool^{nm} \rightarrow \bool^{n^2 - \sqrt{n} - cn}$. Будем обозначать $f^n(x_1 \dots x_n) = f(x_1) \dots f(x_n)$, можно тут же заметить, что $H_1(f_n(\bullet)) = (n-c)n$.

Определим генератор G следующим образом:

$$[h, x_1 \dots x_n]$$
$$\downarrow $$
$$[h, h(f(x_1) \dots f(x_n)), b(x_1) \dots b(x_n)]$$

Необходимо доказать, что
$$[h, h(f^n(\bullet)), b(x_1) \dots b(x_n)] \sim [h, h(f^n(\bullet)), \uniform_n] \sim [h, \uniform_{n^2 - \sqrt{n} - cn}, \uniform_n]$$

Докажем первую эквивалентность. По предыдущей лемме имеем ($r_1 \dots r_n$ - чисто случайные биты)
$$f^n(\bullet)b(x_1) \dots b(x_n) \sim f^n(\bullet) r_1 \dots r_n$$
припишем слева h и применим его к $f_n(\bullet)$. Тогда по свойству III получится как раз требуемая эквивалентность.

Докажем вторую эквивалентность. По лемме о сглаживании имеем $[h, h(x)] \sim [h, \uniform_{n^2 - \sqrt{n} - cn}]$. От замены переменной статистическое расстояние (это то же самое, что L1 расстояние?) (!!!!???? можно ли? ведь f уже не перестановка) не меняется, и от дописывания в конец $n$ случайных битов тоже не меняется. Следовательно, имеем нужную эквивалентность с точностью до $2^{\frac{\sqrt{n}}{2}}$, что является пренебрежимо малой функцией, что нам и нужно.

Длина входа генератора: $|h| + n^2$

Длина выхода генератора: $|h| + n^2 - \sqrt{n} - cn + n$

Видно, что длина входа больше, чем длина выхода.

(дописать про длину x !!!)

Доказано, что $G$ - генератор, (что такое генератор??), у него (

$\Box$

\end{document}